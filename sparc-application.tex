\documentclass[12pt,letter]{article}
\usepackage[margin=1in]{geometry}


\title{My SPARC Application: Technical Question Portion}
\author{Simon Wu}
\date{\today}

\begin{document}
\maketitle

\section{The Question}

I have decided to attempt to answer Question 1A, where the goal is to maximize $\Delta$,
the difference of the end temperatures of the blue and red blocks ($B - R$), after
any number of operations on the 2 blocks. 
Maximizing $\Delta$ is going to require maximizing B and minimizing R.


\section{My Solution}

\subsection{Assumptions and Given Information}

Question 1A provides the following information:
\begin{enumerate}
    \item There are 2 blocks, Blue and Red, with temperatures of $B$ and $R$ 
    respectively, each weighing 1 kilogram.
    \item $B$ begins at $0^{\circ}$C initially and $R$ begins at $100^{\circ}$C
    initially.
    \item Blue and Red possess perfect heat conductivity, and heat equilibrates
    immediately between any chunks of block in contact with each other, based
    on mass.
    \item Each block can be manipulated cut without any heat loss.
\end{enumerate}

For the purposes of my solution, we will also assume that Blue and Red are within
an isolated system where no matter or energy can be exhanged with surroundings.

\subsection{The Solution}

The maximum value of $\Delta$ must be 0, with $B = R$ at the end. This can be 
achieved by touching both blocks together at the beginning and allowing their 
temperatures to equilibrate to $50^{\circ}$C, thus achieving $\Delta$ of 0.

\pagebreak

\subsubsection{Justifications}

Entropy is calculated as heat (energy) absorbed or released by a system or 
surroundings divided by the temperature of the system or surroundings, and is
expressed in Joules per Kelvin ($J/K$).

\bigskip

{\centering
    $\Delta S = Q / T$\par
}

\bigskip

The second law of thermodynamics states that the total entropy of an isolated
system (which we assume is what our blocks are in) can never decrease, and will
remain constant only in reversible processes. 
In any other cases, for example with transferring heat between 2 blocks by contact,
total entropy must always increase.
The second law of thermodynamics can be stated as:

\bigskip

{\centering
    $\Delta S_{universe} = \Delta S_{system} + \Delta S_{surroundings} > 0$\par
}

\bigskip

The second law explains why heat cannot flow from a colder to hotter system cannot
occur without external work. 
For example, if I have 2 blocks named A and B, with A at $10^{\circ}$C and B at
$0^{\circ}$C, and I allow them to make contact, what happens to entropy when some 
energy, say 10 joules, flows from A to B (hot to cold) vs B to A (cold to hot)?

\bigskip

{\centering
    $T_{A}=10^{\circ}$C, $T_{B}=0^{\circ}$C
    
    \bigskip

    $T_{A}=283^{\circ}$K, $T_{B}=273^{\circ}$K

}

\bigskip

\emph{Scenario 1: 10 joules flows from B to A}

\bigskip

{\centering
    $Q_A = 10$ J, $Q_B = -10$ J

    \bigskip

    $\Delta S = Q_A/T_A + Q_B/T_B$

    \bigskip

    $\Delta S = 10/283 - 10/273$

    \bigskip

    $\Delta S = -0.00129\ J/K$ 

}

\bigskip

This is impossible: entropy can never decrease in an isolated system without external 
work.

\bigskip

\emph{Scenario 2: 5 joules flows from A to B}

\bigskip

{\centering
    $Q_A = -10$ J, $Q_B = 10$ J

    \bigskip

    $\Delta S = Q_A/T_A + Q_B/T_B$

    \bigskip

    $\Delta S = -10/283 + 10/273$

    \bigskip

    $\Delta S = 0.00129\ J/K$ 

}

\bigskip

This is possible, as it does not break the second law of thermodynamics.

\bigskip

Why is this relevant to Question 1A?
The second law dictates how heat can flow between objects.
Because entropy always increases in an isolated system and never decreases, we can
start to make some judgements on what the bounds of the end temperatures of B and R
are going to be.

\pagebreak

Because of the situation I described earlier, no heat will be able to flow from B
to R, as B starts colder than R, therefore the minimum temperature of B is $0^{\circ}$C
and the maximum temperature of R is $100^{\circ}$C.
Additionally, because of that same 


\end{document}